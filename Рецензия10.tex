\documentclass[12pt,twoside]{article}
\usepackage{mathtext}
\usepackage[T2A]{fontenc}
\usepackage[utf8]{inputenc}
\usepackage[english, russian]{babel}
\usepackage[pdftex]{graphicx,rotating}
\usepackage{amssymb}
\usepackage{amsthm}
\usepackage{bm}
\usepackage{color}
\usepackage{latexsym}
\usepackage{titlesec}
\usepackage{amsmath}
\usepackage{amsfonts}
\usepackage{cite}
\usepackage{indentfirst}
\usepackage{enumitem}
\setlist{noitemsep}
\usepackage{cmap}
\usepackage[a4paper, left=25mm, right=20mm, bottom=20mm, top=20mm]{geometry}
\frenchspacing
\pagestyle{plain}
\setlength{\parindent}{0cm}

\begin{document}
\centerline{\bf\large Писменная рецензия команды ЛНМО-7}	
\centerline{\bf\large на задачу\textnumero10 ``Циферки''}
\centerline{\bf\large команды гимназии \textnumero61 г.Минска}
\vspace{12pt}

\subsection*{Резюме по итогам проведённого исследования}
\noindent Пункты 1 и 2 решены полностью верно. Пункт 3 имеет правильный ответ, но неверное решение. Пункт 4 не решен. В пункте 5 приведён ответ для одного конкретного \linebreak $N=\NOK(2,3,\ldots,18)$. Настоящих обобщений не предоставлено.
	
\subsection*{Нелепицы}
\begin{itemize}
\item В пункте 3 Автор сомневается в полученном ответе, хотя он верный. Сомневается она не зря, ведь доказательство отсутствует, а рассуждения приведены на уровне идей.
	
\item Перебор таблиц остатков можно было существенно сократить. Для того, чтобы найти остатки при делении на 23 чисел вида $n\cdot10^k$, автор указывает остатки для каждого числа, вместо того, чтобы умножить на $n$ остаток от деления $10^k$ на 23.

\item В пункте 5 была поставлена задача исследовать пункт 4 для произвольного натурального числа. Однако в решении указаны признаки делимости на простые числа. То есть, на вопрос пункта предоставленное решение не отвечает. Но в авторском обобщении, они приводят наименьшее число в записи которого все 10 цифр, которое делится на $12\;252;240=\NOK(2,3,\ldots,18)$, хотя и не приводят строгих доказательств.
\end{itemize}

\subsection*{Недочёты и Опечатки}
\paragraph{Пункт 3} В решении присутствует недочёт. Автор говорит, что сумма цифр стоящих на четных и нечетных местах это 28 и 17, но он не указывает почему это так.

Следовало разобрать несколько случаев:

Пусть $x$ и $y$ ~--- это сумма цифр на нечётных и чётных местах, тогда кроме системы, $\begin{cases}
	x+y=45,\\
	x-y=11,
\end{cases}$ рассмотренной автором, надо было рассмотреть и другие варианты:
$\begin{cases}
	x+y=45,\\
	x-y=-11;
\end{cases}$
$\begin{cases}
	x+y=45,\\
	x-y=22;
\end{cases}$ \ldots


\subsection*{Качественная оценка исследования}
Автором была проделана хорошая работа по пунктам 1, 2 и 6. Однако в остальных пунктах присутствуют как ошибки, так и недочеты. И в целом следует признать, что работа авторами выполнена \textit{халтурно}, а исследование проведено  \textit{неудовлетворительным} образом.
\end{document}